\documentclass[handout]{beamer}

\usepackage[utf8]{inputenc}
\usepackage[english,russian]{babel}
\usepackage{listings}

%\ifx\pdftexversion\undefined
%\usepackage[dvips]{graphicx}
%\else
%\usepackage[pdftex]{graphicx}
%\DeclareGraphicsRule{*}{mps}{*}{}
%#\fi

\definecolor{javared}{rgb}{0.6,0,0} % for strings
\definecolor{javagreen}{rgb}{0.25,0.5,0.35} % comments
\definecolor{javapurple}{rgb}{0.5,0,0.35} % keywords
\definecolor{javadocblue}{rgb}{0.25,0.35,0.75} % javadoc
 
\lstset{language=Java,
basicstyle=\footnotesize\ttfamily,
keywordstyle=\color{javapurple}\bfseries,
stringstyle=\color{javared},
commentstyle=\color{javagreen},
morecomment=[s][\color{javadocblue}]{/**}{*/},
stepnumber=2,
numbersep=10pt,
tabsize=4,
showspaces=false,
showstringspaces=false}


\title{Паттерн Strategy}
\author{Есилевич Александр}

\begin{document}

\maketitle

\begin{frame}[fragile]
\frametitle{Инструмент соединения фигур}
\begin{itemize}
\item Пользователь выбирает инструмент соединения
\item Каждая фигура имеет набор соединительных точек
\item При наведении на соединительную точку она подсвечивается
\item После клика на точке пользователь тянет линию к другой соединительной точке
\item Соединительные линии перемещаются при изменении/перемещении фигур
\end{itemize}
\end{frame}


\begin{frame}[fragile]
\frametitle{Инструмент соединения фигур}
Существенное требование: доступно несколько вариантов соединительных линий, пользователь
может выбирать и изменять тип соединительной линии:
     \begin{itemize}
     \item Прямые линии
     \item Ломаные линии, идущие по вертикали и горизонтали
     \item Кривые линии
     \item Линии, огибающие фигуры
     \end{itemize}
\end{frame}


\begin{frame}[fragile]
\frametitle{Реализация}
\begin{itemize}
\item Для идентификации соединительных точек создадим класс \lstinline{ConnectPoint},
      по аналогии с классом \lstinline{SelectPoint}
\item Для определения точек соединения для каждой фигуры используем паттерн Visitor
\item Каждая соединительная линия - тоже фигура, которую можно выделять и перемещать,
      поэтому создадим класс \lstinline{ConnectorShape}, унаследованный от класса
      \lstinline{Shape}
\end{itemize}
\end{frame}


\begin{frame}[fragile]
\frametitle{Возникающие проблемы}
\begin{itemize}
\item Как реализовать изменение типа соединительной линии?
\item Для разных типов соединительных линий меняется алгоритм отрисовки и выделения,
      но все типы соединительных линий содержат ссылки на два объекта
      класса \lstinline{ConnectPoint}, и алгоритм перемещения точек соединения одинаковый.
      Как избежать дублирования кода?
\item Алгоритм отрисовки используется не только в нарисованной соединительной линии,
      но и при её рисовании. Как избежать дублирования кода?
\end{itemize}
\end{frame}


\begin{frame}[fragile]
\frametitle{Вариант реализации}
Сделаем класс \lstinline{ConnectorShape} абстрактным, и унаследуем от него
      отдельный класс для каждой соединительной линии
\end{frame}



\begin{frame}[fragile]
\frametitle{Недостатки}
\begin{itemize}
\item Как изменять стратегию отрисовки у существующей соединительной линии? Это не удобно --
      инструменту изменения свойств требуется удалить линию из списка фигур, и создать другую
\item Как переиспользовать алгоритм отрисовки при рисовании линии?
\end{itemize}
\end{frame}



\begin{frame}[fragile]
\frametitle{Решение}
Паттерн Strategy -- отдедяет семейство алгоритмов, инкапсулирует каждый из них, и делает
из взаимозаменяемыми
\end{frame}


\begin{frame}[fragile]
\frametitle{Диаграмма классов}
\begin{center}
\includegraphics{strategy.mps}
\end{center}
\end{frame}



\begin{frame}[fragile]
\frametitle{Диаграмма классов для ConnectStrategy}
\begin{center}
\includegraphics{connect_strategy.mps}
\end{center}
\end{frame}


\begin{frame}[fragile]
\frametitle{Результаты}
\begin{itemize}
\item Нет необходимости в громоздских условных операторах
\item Нет необходимости порождать новые классы для каждой пары Context + алгоритм
\item Алогоритм может быть переиспользован в разных контекстах
\end{itemize}
\end{frame}



\end{document}
