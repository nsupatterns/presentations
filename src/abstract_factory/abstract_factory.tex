\documentclass[handout]{beamer}

\usepackage[utf8]{inputenc}
\usepackage[english,russian]{babel}
\usepackage{listings}

%\ifx\pdftexversion\undefined
%\usepackage[dvips]{graphicx}
%\else
%\usepackage[pdftex]{graphicx}
%\DeclareGraphicsRule{*}{mps}{*}{}
%#\fi

\definecolor{javared}{rgb}{0.6,0,0} % for strings
\definecolor{javagreen}{rgb}{0.25,0.5,0.35} % comments
\definecolor{javapurple}{rgb}{0.5,0,0.35} % keywords
\definecolor{javadocblue}{rgb}{0.25,0.35,0.75} % javadoc
 
\lstset{language=Java,
basicstyle=\footnotesize\ttfamily,
keywordstyle=\color{javapurple}\bfseries,
stringstyle=\color{javared},
commentstyle=\color{javagreen},
morecomment=[s][\color{javadocblue}]{/**}{*/},
stepnumber=2,
numbersep=10pt,
tabsize=4,
showspaces=false,
showstringspaces=false}


\title{Паттерн Abstract Factory}
\author{Есилевич Александр}

\begin{document}

\maketitle

\begin{frame}[fragile]
\frametitle{Поддержка нескольких стилей графического интерфейса}
Хочется добиться следующего:
\begin{itemize}
\item Приложение должно поддерживать несколько стилей графического интерфейса (обозначим их Style1 и Style2);
\item Код приложения, отвечающий за графический интерфейс, не должен дублироваться для каждого стиля.
\end{itemize}
\end{frame}


\begin{frame}[fragile]
\frametitle{Зафиксируем набор графических компонентов, используемых в приложении}
Определим для каждого компонента и стиля подкласс
\vspace{1cm}

\begin{tabular}{|l|l|l|}
\hline
Компонент		&	Style1					&	Style2 	\\
\hline
Окно			&	Style1Window		& 	Style2Window   	\\
Кнопка			&	Style1Button		&	Style2Button		\\
Текстовое поле	&	Style1EditBox		&	Style2EditBox		\\
Полоса прокрутки	&	Style1ScrollBar	&	Style2ScrollBar	\\
\hline
\end{tabular}
\end{frame}


\begin{frame}[fragile]
\frametitle{Диаграмма классов графических компонентов}
\begin{center}
\includegraphics{components.mps}
\end{center}
\end{frame}


\begin{frame}[fragile]
\begin{center}
Как сделать код приложения независимым от конкретного набора инструментов?
\end{center}
\end{frame}


\begin{frame}[fragile]
\frametitle{Паттерн AbstractFactory}
Назначение: предоставляет интерфейс для создания семейств взаимосвязанных объектов, не определяя конкретных классов
этих объектов.
\end{frame}


\begin{frame}[fragile]
\frametitle{Диаграмма классов с использованием AbstractFactory}
\begin{center}
\includegraphics{acomponents.mps}
\end{center}
\end{frame}


\begin{frame}[fragile]
\frametitle{Диаграмма классов AbstractFactory}
\begin{center}
\includegraphics{abstract_factory.mps}
\end{center}
\end{frame}


\begin{frame}[fragile]
\frametitle{Достоинства}
\begin{itemize}
\item Изолирует конкретные классы от клиента
\item Упрощает замену семейства продуктов
\item Гарантирует сочитаемость продуктов
\end{itemize}
\end{frame}


\begin{frame}[fragile]
\frametitle{Недостатки}
\begin{itemize}
\item Поддержать новый вид продуктов трудно
\end{itemize}
\end{frame}

\end{document}
