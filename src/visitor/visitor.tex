\documentclass[handout]{beamer}

\usepackage[utf8]{inputenc}
\usepackage[english,russian]{babel}
\usepackage{listings}

%\ifx\pdftexversion\undefined
%\usepackage[dvips]{graphicx}
%\else
%\usepackage[pdftex]{graphicx}
%\DeclareGraphicsRule{*}{mps}{*}{}
%#\fi

\definecolor{javared}{rgb}{0.6,0,0} % for strings
\definecolor{javagreen}{rgb}{0.25,0.5,0.35} % comments
\definecolor{javapurple}{rgb}{0.5,0,0.35} % keywords
\definecolor{javadocblue}{rgb}{0.25,0.35,0.75} % javadoc
 
\lstset{language=Java,
basicstyle=\footnotesize\ttfamily,
keywordstyle=\color{javapurple}\bfseries,
stringstyle=\color{javared},
commentstyle=\color{javagreen},
morecomment=[s][\color{javadocblue}]{/**}{*/},
stepnumber=2,
numbersep=10pt,
tabsize=4,
showspaces=false,
showstringspaces=false}


\title{Паттерн Visitor}
\author{Есилевич Александр}

\begin{document}

\maketitle

\begin{frame}[fragile]
\frametitle{Выделение фигур}
\begin{itemize}
\item Пользователь кликает мышкой на фигуру
\item Появляются точки выделения фигуры
\item Пользователь либо перемещает точки выделения, либо перемещает всю фигуру
\end{itemize}
\end{frame}


\begin{frame}[fragile]
\frametitle{Вопросы реализации}
\begin{itemize}
\item Как определить, что кликом в определённой точке выбирается какая-то фигура?
\item Как определить какие точки выделения есть у объекта класса \lstinline{Shape}?
\end{itemize}
\end{frame}


\begin{frame}[fragile]
\frametitle{Вопросы реализации}
\begin{itemize}
\item Для каждой конкретной фигуры можно определить область выделения
\item Для каждой конкретной фигуры известны точки выделения
\end{itemize}

\vspace{1cm}

Можно расширить интерфейс класа \lstinline{Shape} так, чтобы каждый объект
этого класса умел определять, выделен ли он кликом в заданной точке, и 
умел возвращать список точек выделения
\end{frame}


\begin{frame}[fragile]
\frametitle{Класс SelectPoint}
\begin{lstlisting}
public abstract class SelectPoint {
    public abstract Point getPos();
    public abstract void setPos(Point newPos);
}
\end{lstlisting}
\end{frame}


\begin{frame}[fragile]
\frametitle{Расширенный интерфейс класса Shape}
\begin{lstlisting}
public class Shape {
    ...
    
    public abstract boolean trySelect(int x, int y);
    public abstract int getSelectPointsCount();
    public abstract SelectPoint getSelectPoint(int index);
}
\end{lstlisting}
\end{frame}


\begin{frame}[fragile]
\frametitle{Реализация класса SelectTool}
\begin{itemize}
\item Четыре состояния: Started, Selected, Move, MovePoint
\item В начальном состоянии при клике SelectTool вызывает \lstinline{trySelect} для всех
      фигур и определяет выбранную
\item В состоянии Selected для отрисовки вызывается \lstinline{getSelectPoint} для определния
      положения точек выделения. При нажатии мыши по выделенной фигуре осуществляется переход
      в состояние Move, При клике на точке -- переходв в состояние MovePoint
\item При движении мыши в состоянии MovePoint у выбранной точки выделения меняется положение,
      точка выделения меняет всю фигуру.
\end{itemize}
\end{frame}


\begin{frame}[fragile]
\frametitle{Реализация точек редактирования для линии}
\begin{lstlisting}
class LineStartSelectPoint extends SelectPoint {
    public LineStartSelectPoint(Line l) {
        line = l;
    }

    public Point getPos() {
        return line.getStart();
    }

    public void setPos(Point newPos) {
        line.setStart(newPos);
    }

    private Line line;
}
\end{lstlisting}
\end{frame}


\begin{frame}[fragile]
\frametitle{Реализация точек редактирования для линии}
\begin{lstlisting}
class LineEndSelectPoint extends SelectPoint {
    public LineEndSelectPoint(Line l) {
        line = l;
    }

    public Point getPos() {
        return line.getEnd();
    }

    public void setPos(Point newPos) {
        line.setEnd(newPos);
    }

    private Line line;
}
\end{lstlisting}
\end{frame}


\begin{frame}[fragile]
\frametitle{Реализация точек редактирования для линии}
\begin{lstlisting}
class Line extends Shape {
    ...
    
    public int getSelectPointsCount() {
        return 2;
    }

    public SelectPoint getSelectPoint(int index) {
        if(index == 0)
            return new LineStartSelectPoint(this);
        else
            return new LineEndSelectPoint(this);
    }

    ...
}
\end{lstlisting}
\end{frame}


\begin{frame}[fragile]
\frametitle{Недостатки}
\begin{itemize}
\item В класс \lstinline{Shape} добавились ''лишние'' методы, которые относятся к конкретному
      инструменту, а не к фигуре
\item При увеличении количества инструментов, зависящих от конкретных фигур, интерфейс
      класса \lstinline{Shape} будет сильно разрастаться (пример инструмента - линии-коннекторы)
\end{itemize}
\end{frame}


\begin{frame}[fragile]
\frametitle{Ключевая проблема}
Нужен универсальный способ перебора всех фигур в списке и выполнения определённых действий
в зависимости от типа фигуры.
\end{frame}


\begin{frame}[fragile]
\frametitle{Решение}
Паттерн Visitor -- описывает операцию, выполняемую с каждым оъектом некоторой структуры,
при этом позволяя определить новую операцию не меняя классы объектов
\end{frame}


\begin{frame}[fragile]
\frametitle{Применимость}
\begin{itemize}
\item Присутствуют объекты многих классов и требуется выполнять над ними операции, 
      зависящие от конкретных классов
\item Выполняемые операции не связаны между собой, поэтому не хочется засорять
      ими базовый класс всех объектов
\item Классы изменяются редко, а операции добавляются часто
\end{itemize}
\end{frame}


\begin{frame}[fragile]
\frametitle{Диаграмма классов}
\begin{center}
\includegraphics{visitor.mps}
\end{center}
\end{frame}


\begin{frame}[fragile]
\frametitle{Диаграмма классов для ShapeVisitor}
\begin{center}
\includegraphics{shape_visitor.mps}
\end{center}
\end{frame}


\begin{frame}[fragile]
\frametitle{Реализация SelectPointsCollector для линии}
\begin{lstlisting}
class SelectPointsCollector extends ShapeVisitor {
    public SelectPointsCollector(Point p) {
        pos = p;
    }
    
    public void visitLine(Line shape) {
        points.add(new LineStartSelectPoint(shape));
        points.add(new LineEndSelectPoint(shape));
    }
    
    public SelectPointsList getPoints() { return points; }

    private Point pos;
    private SelectPointsList points;
}
\end{lstlisting}
\end{frame}


\begin{frame}[fragile]
\frametitle{Преимущества}
\begin{itemize}
\item Упрощает добавление новых операций
\item Объединяет родственные операции и отедяет друг от друга операции, не
      имеющие отношения друг к другу
\item Позволяет посещать различные структуры, не обязательно однородные
\end{itemize}
\end{frame}



\begin{frame}[fragile]
\frametitle{Недостатки}
\begin{itemize}
\item Затрудняет добавление новых классов в иерархию
\end{itemize}
\end{frame}


\end{document}
