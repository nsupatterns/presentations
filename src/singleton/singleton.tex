\documentclass[handout]{beamer}

\usepackage[utf8]{inputenc}
\usepackage[english,russian]{babel}
\usepackage{listings}

%\ifx\pdftexversion\undefined
%\usepackage[dvips]{graphicx}
%\else
%\usepackage[pdftex]{graphicx}
%\DeclareGraphicsRule{*}{mps}{*}{}
%#\fi

\definecolor{javared}{rgb}{0.6,0,0} % for strings
\definecolor{javagreen}{rgb}{0.25,0.5,0.35} % comments
\definecolor{javapurple}{rgb}{0.5,0,0.35} % keywords
\definecolor{javadocblue}{rgb}{0.25,0.35,0.75} % javadoc
 
\lstset{language=Java,
basicstyle=\footnotesize\ttfamily,
keywordstyle=\color{javapurple}\bfseries,
stringstyle=\color{javared},
commentstyle=\color{javagreen},
morecomment=[s][\color{javadocblue}]{/**}{*/},
stepnumber=2,
numbersep=10pt,
tabsize=4,
showspaces=false,
showstringspaces=false}


\title{Паттерн Singleton}
\author{Есилевич Александр}

\begin{document}

\maketitle

\begin{frame}[fragile]
\frametitle{Назначение}
Гарантирует, что у класс есть только один экземпляр, и предоставляет к нему
глобальную точку доступа
\end{frame}


\begin{frame}[fragile]
\frametitle{Примеры}
\begin{itemize}
\item файловая система
\item журнал приложения
\item оконный менеджер (графическая система)
\end{itemize}
\end{frame}


\begin{frame}[fragile]
\frametitle{Реализация с помощью глобальной переменной}
\begin{itemize}
\item Даёт глобальный доступ к объекту
\item Не запрещает создавать другие экземпляры класса
\end{itemize}
\end{frame}


\begin{frame}[fragile]
\frametitle{Паттерн Singleton}
\begin{lstlisting}
class Singleton {
    public static instance() {
        if(inst == null) {
            // create instance
        }
        
        return inst;
    }
    
    // constructor is not public
    protected Singleton() {}
    
    // public singleton operations
    
    private static Singleton inst;
}
\end{lstlisting}
\end{frame}


\begin{frame}[fragile]
\frametitle{Пример использования: Abstract Factory}
\begin{lstlisting}
class ComponentFactory {
    public static instance() {
        if(inst == null) {
            String style = readStyleFromConfig();
            if(style == "style1")
                inst = new Style1Factory();
            else
                inst = new Style2Factory();
        }
        
        return inst;
    }
    
    protected ComponentFactory() {}
    
    public abstract SystemButton createButton();
    public abstract SystemEditBox createEditBox();
    
    private static ComponentFactory inst;
}
\end{lstlisting}
\end{frame}


\begin{frame}[fragile]
\frametitle{Пример использования: Abstract Factory}
\begin{lstlisting}
SystemButton button1 =
    ComponentFactory.instance().createButton();
button1.setListener(...);

SystemButton button2 =
    ComponentFactory.instance().createButton();
button2.setListener(...);
\end{lstlisting}
\end{frame}



\begin{frame}[fragile]
\frametitle{Проблемы реализации}
Как сделать так, чтобы наследников \lstinline{ComponentFactory} мог создавать
только сам \lstinline{ComponentFactory}?
\begin{itemize}
\item В С++ -- сделать конструкторы наследников закрытыми и объявить \lstinline{ComponentFactory} другом;
\item В Java -- использовать пакеты и использовать package-private модификатор доступа
\item Сделать конструкторы наследников закрытыми, определить метод \lstinline{instance} в каждом из них,
      и вызывать эти методы в \lstinline{ComponentFactory.instance}.
\end{itemize}
\end{frame}




\begin{frame}[fragile]
\frametitle{Результаты}
\begin{itemize}
\item контролируемый доступ к единственному экземпляру
\item не засоряет пространство имён глобальными переменными
\item допускает уточнение операций с помощью подклассов
\item допускает переменное число экземпляров
\end{itemize}
\end{frame}



\end{document}
