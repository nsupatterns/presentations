\documentclass[handout]{beamer}

\usepackage[utf8]{inputenc}
\usepackage[english,russian]{babel}
\usepackage{listings}

%\ifx\pdftexversion\undefined
%\usepackage[dvips]{graphicx}
%\else
%\usepackage[pdftex]{graphicx}
%\DeclareGraphicsRule{*}{mps}{*}{}
%#\fi

\definecolor{javared}{rgb}{0.6,0,0} % for strings
\definecolor{javagreen}{rgb}{0.25,0.5,0.35} % comments
\definecolor{javapurple}{rgb}{0.5,0,0.35} % keywords
\definecolor{javadocblue}{rgb}{0.25,0.35,0.75} % javadoc
 
\lstset{language=Java,
basicstyle=\footnotesize\ttfamily,
keywordstyle=\color{javapurple}\bfseries,
stringstyle=\color{javared},
commentstyle=\color{javagreen},
morecomment=[s][\color{javadocblue}]{/**}{*/},
stepnumber=2,
numbersep=10pt,
tabsize=4,
showspaces=false,
showstringspaces=false}


\title{Паттерн State}
\author{Есилевич Александр}

\begin{document}

\maketitle

\begin{frame}[fragile]
\frametitle{Интерактивное рисование фигур}
\begin{itemize}
\item Пользователь выбирает инструмент в панели
\item Пользователь выполняет манипуляции мышью/клавиатурой на изображении
\item В изображении выделяются фигуры или добавляются новые
\end{itemize}
\end{frame}


\begin{frame}[fragile]
\frametitle{Первая реализация}
\begin{itemize}
\item В классе \lstinline{Scene} храним ID выбранного инструмента
\item Обрабатываем события мыши/клавиатуры тем или иным образом в
      зависмиости от ID выбранного инструмента
\end{itemize}
\end{frame}


\begin{frame}[fragile]
\frametitle{Первая реализация}
\begin{lstlisting}
enum ToolId { TOOL_SELECT, TOOL_LINE,
              TOOL_RECTANGLE, TOOL_CIRCLE }
ToolId selectedToolId;

void onMouseDown(int x, int y) {
    switch(selectedToolId) {
    case TOOL_LINE:
        // line drawing logic
        break;
    case ...
    }
}

void onMouseUp(int x, int y) { ... }
void onMouseMove(int x, int y) { .. }
\end{lstlisting}
\end{frame}


\begin{frame}[fragile]
\frametitle{Как можно улучшить?}
\begin{itemize}
\item Логика обработки каждого инструмента разная
\item Инструменты не зависят друг от друга
\item Инструменты обрабатывают одни и те же действия пользователя
\end{itemize}
\vspace{1cm}
Вывод: можно создать абстрактный класс \lstinline{Tool},
определяющий интерфейс обработки действий пользователя,
и опеделить класс-наследник для каждого инструмента
\end{frame}


\begin{frame}[fragile]
\frametitle{Класс Tool и наследники}
\begin{center}
\includegraphics{tools.mps}
\end{center}
\end{frame}


\begin{frame}[fragile]
\frametitle{Реализация класса LineTool}
Есть три состояния:
\begin{itemize}
\item \lstinline{START} -- рисование только начали
\item \lstinline{PRESSED} -- нажали кнопку мыши, ''тянем'' линию, пока
   	  кнопку не отпустят
\item \lstinline{FREE_MOVE} -- кнопку мыши отпустили там же, где нажали,
   	 ''тянем'' линию, пока кнопку ещё раз не нажмут
\end{itemize}
\vspace{1cm}
Класс \lstinline{LineTool} хранит ссылку на список фигур, состояние
рисования, начальные и конечные координаты линии.
\end{frame}


\begin{frame}[fragile]
\frametitle{Реализация класса LineTool}
\begin{lstlisting}
public class LineTool extends Tool {
    enum State { START, PRESSED, FREE_MOVE }
    
    private ShapeList shapes;
    private State state;
    private int startX, startY, endX, endY;
    
    public void onMouseDown() { ... }
    public void onMouseUp() { ... }
    public void onMouseMove() { ... }
}
\end{lstlisting}
\end{frame}


\begin{frame}[fragile]
\frametitle{Реализация класса LineTool}
\begin{lstlisting}
public void onMouseDown(int x, int y) {
    switch(state) {
    case START:
        startX = x;
        startY = y;
        endX = x;
        endY = y;
        state = State.PRESSED;
        break;
    case PRESSED:
        break;
    case FREE_MOVE:
        break;
    }
}
\end{lstlisting}
\end{frame}


\begin{frame}[fragile]
\frametitle{Реализация класса LineTool}
\begin{lstlisting}
public void onMouseMove(int x, int y) {
    switch(state) {
    case START:
        break;

    case PRESSED:
    case FREE_MOVE:
        endX = x;
        endY = y;
        break;
    }
    endX = x;
    endY = y;
}
\end{lstlisting}
\end{frame}


\begin{frame}[fragile]
\frametitle{Реализация класса LineTool}
\begin{lstlisting}
public void onMouseUp(int x, int y) {
    switch(state) {
    case START:
        break;

    case PRESSED:
        if(startX == x && startY == y) {
            state = State.FREE_MOVE;
        } else {
            shapes.add(new Line(startX, startY, x, y));
            state = State.START;
        }
        break;

    case FREE_MOVE:
        shapes.add(new Line(startX, startY, x, y));
        state = State.START;
        break;
    }
}
\end{lstlisting}
\end{frame}


\begin{frame}[fragile]
\frametitle{Реализация класса LineTool}
\begin{lstlisting}
public void draw(SystemCanvas canvas) {

    if(state == State.START) {
        return;
    }

    (new Line(startX, startY, endX, endY)).draw(canvas);
}
\end{lstlisting}
\end{frame}


\begin{frame}[fragile]
\frametitle{Недостатки реализации LineTool}
\begin{itemize}
\item Присутствуют большие операторы \lstinline{switch}
\item Логика перехода между состояниями явно не выделена
\item Код сложен для понимания и добавления новой функциональности
\end{itemize}
\end{frame}


\begin{frame}[fragile]
\begin{center}
Решение: ввести отдельный класс для каждого состояния
\end{center}
\end{frame}


\begin{frame}[fragile]
\frametitle{Диаграмма классов LineToolState}
\begin{center}
\includegraphics{linetool.mps}
\end{center}
\end{frame}


\begin{frame}[fragile]
\frametitle{Кто будет изменять состояние?}
\begin{itemize}
\item класс \lstinline{LineTool}
\item каждое состояние будет само изменять состояние
    \begin{itemize}
    \item классы состояний меют доступ к \lstinline{LineTool}
    \item обработчики пользовательского ввода будут возвращают
          новое состояние
    \end{itemize}
\end{itemize}
\end{frame}


\begin{frame}[fragile]
\frametitle{Модифицированный класс LineTool}
\begin{lstlisting}
public class LineTool extends Tool {
    LineTool(ShapeList list) {
        shapes = list;
        state = new LineToolStateStart(shapes);
    }
    public void draw(SystemCanvas canvas) {
        state.draw(canvas); }
    public void reset() {
        state = new LineToolStateStart(shapes) ]
    }
    public void onMouseDown(int x, int y) {
        state = state.onMouseDown(x, y);
    }
    public void onMouseUp(int x, int y) {
        state = state.onMouseUp(x, y);
    }
    public void onMouseMove(int x, int y) {
        state = state.onMouseMove(x, y);
    }
    ShapeList shapes;
    LineToolState state;
}
\end{lstlisting}
\end{frame}


\begin{frame}[fragile]
\frametitle{Класс LineToolStateStart}
\begin{lstlisting}
class LineToolStateStart extends LineToolState {
    public LineToolStateStart(ShapeList list) {
        shapes = list;
    }

    public void draw(java.awt.Graphics canvas) {}

    public LineToolState onMouseDown(int x, int y) {
        return new LineToolStatePressed(shapes, x, y);
    }

    public LineToolState onMouseUp(int x, int y) {
    	return this;
    }
    
    public LineToolState onMouseMove(int x, int y) {
        return this;
    }

    private ShapeList shapes;
};
\end{lstlisting}
\end{frame}


\begin{frame}[fragile]
\frametitle{Класс LineToolStatePressed}
\begin{lstlisting}
    private ShapeList shapes;
    private int startX, startY, endX, endY;
    
    public LineToolStatePressed(ShapeList list,
                                int x, int y) {
        shapes = list;
        startX = x;
        startY = y;
        endX = x;
        endY = y;
    }
    
    public void draw(java.awt.Graphics canvas) {
        (new Line(startX, startY, endX, endY))
            .draw(canvas);
    }
    
    public LineToolState onMouseDown(int x, int y) {
        return this;
    }
\end{lstlisting}
\end{frame}
    
\begin{frame}[fragile]
\frametitle{Класс LineToolStatePressed}
\begin{lstlisting}      
    public LineToolState onMouseUp(int x, int y) {
        if(x == startX && y == startY) {
            return new LineToolStateFreeMove(shapes, x, y);
        } else {
            shapes.add(new Line(startX, startY, x, y));
            return new LineToolStateStart(shapes);
        }
    }
    
    public LineToolState onMouseMove(int x, int y) {
        endX = x; endY = y; return this;
    }
\end{lstlisting}
\end{frame}



\begin{frame}[fragile]
\frametitle{Назначение}
Паттерн State позволяет объекту изменять своё поведение в зависимости
от внутреннего состояния. Извне создаётся впечатление, что изменился
класс объекта.
\end{frame}


\begin{frame}[fragile]
\frametitle{Диаграмма классов}
\begin{center}
\includegraphics{state.mps}
\end{center}
\end{frame}


\begin{frame}[fragile]
\frametitle{Результаты}
Преимущества:
\begin{itemize}
\item Локализует зависящее от состояния поведение и делит его
      на части, соответствующие состояниям
\item Избавляет от необходимости использования больших операторов
      \lstinline{if} и \lstinline{switch}
\item Делает явными переходы между состояниями
\item Объекты состояния можно разделять
\end{itemize}

\vspace{1cm}

Недостатки:
\begin{itemize}
\item Увеличивает число классов
\end{itemize}
\end{frame}


\begin{frame}[fragile]
\frametitle{Класс Tool тоже может быть паттерном State}
\begin{itemize}
\item Определяем один класс \lstinline{ToolState} для всех инструментов
\item Задаём начальное состояние при выборе инструмента
\item В качестве контекста в данном случае выступает класс \lstinline{Scene}
\end{itemize}
\end{frame}


\begin{frame}[fragile]
\frametitle{Пример: TcpConnection}
\begin{lstlisting}
public class TcpConnection {
    public bool connect(String addr, int port);
    public void close();    
    public void send(byte [] data);
    public byte [] recv();
    public void listen(int port);
    public void accept();
}
\end{lstlisting}
\end{frame}


\begin{frame}[fragile]
\frametitle{Пример: TcpConnection}
\begin{lstlisting}
public class TcpConnection {
    private enum State { CLOSED, CONNECTED, LISTEN }
    private State state;
    
    TcpConnection() {
        state = CLOSED;
    }
    
    public bool connect(String addr, int port) {
        if (state == CONNECTED || state == LISTEN) {
            // throw exception
        }
        
        if (connectToRemoteHost(addr, port)) {
            state = CONNECTED;
            return true;
        }
        return false;
    }
}
\end{lstlisting}
\end{frame}


\begin{frame}[fragile]
\frametitle{Пример: TcpConnection}
\begin{lstlisting}
public class TcpConnection {
    public void send(byte [] data) {
        if (state != CONNECTED) {
            // throw exception
        }
        
        // send data
    }
    
    public void close() {
        if (state == LISTEN) {
            // cancel listening
        } else if(state == CONNECTED) {
            // send disconnect packet
        } else {
            // throw exception
        }
        
        state = CLOSED;
    }
}
\end{lstlisting}
\end{frame}


\begin{frame}[fragile]
\frametitle{Пример: TcpConnection}
\begin{center}
\includegraphics{tcpconnection.mps}
\end{center}
\end{frame}


\begin{frame}[fragile]
\frametitle{Пример: TcpConnection}
\begin{lstlisting}
public class TcpClosed {
    public bool connect(String addr, int port) {
        if (connectToRemoteHost(addr, port)) {
            changeState(new TcpConnected());
            return true;
        }
        return false;
    }
    
    public void listen(int port) {
        changeState(new TcpListen());
    }
    
    public void send() { /* throw exception */ }
    public void recv() { /* throw exception */ }    
    public void close() { /* throw exception */ }
    public void accept() { /* throw exception */ }
}
\end{lstlisting}
\end{frame}


\begin{frame}[fragile]
\frametitle{Пример: TcpConnection}
\begin{lstlisting}
public class TcpListen {   
    public void close() {
        // stop listening
        changeState(new TcpClosed());
    }
    
    public void accept() {
        // accept connection
        changeState(new TcpConnected());
    }
    
    public bool connect() { /* throw exception */ }
    public void send() { /* throw exception */ }
    public void recv() { /* throw exception */ }
    public void listen() { /* throw exception */ }
}
\end{lstlisting}
\end{frame}


\begin{frame}[fragile]
\frametitle{Пример: TcpConnection}
\begin{lstlisting}
public class TcpConnected {  
    public void close() {
        // close connection
        changeState(new TcpClose());
    }
    
    public void send(byte [] data) {
        // send data
    }

    public byte [] recv() {
        // receive data
    }
    
    public bool connect() { /* throw exception */ }
    public void listen() { /* throw exception */ }
    public void accept() { /* throw exception */ }
}
\end{lstlisting}
\end{frame}



\end{document}
