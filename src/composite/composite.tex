\documentclass[handout]{beamer}

\usepackage[utf8]{inputenc}
\usepackage[english,russian]{babel}
\usepackage{listings}

%\ifx\pdftexversion\undefined
%\usepackage[dvips]{graphicx}
%\else
%\usepackage[pdftex]{graphicx}
%\DeclareGraphicsRule{*}{mps}{*}{}
%#\fi

\definecolor{javared}{rgb}{0.6,0,0} % for strings
\definecolor{javagreen}{rgb}{0.25,0.5,0.35} % comments
\definecolor{javapurple}{rgb}{0.5,0,0.35} % keywords
\definecolor{javadocblue}{rgb}{0.25,0.35,0.75} % javadoc
 
\lstset{language=Java,
basicstyle=\footnotesize\ttfamily,
keywordstyle=\color{javapurple}\bfseries,
stringstyle=\color{javared},
commentstyle=\color{javagreen},
morecomment=[s][\color{javadocblue}]{/**}{*/},
stepnumber=2,
numbersep=10pt,
tabsize=4,
showspaces=false,
showstringspaces=false}


\title{Паттерн Composite}
\author{Есилевич Александр}

\begin{document}

\maketitle

\begin{frame}[fragile]
\frametitle{Перемещение фигур}
\begin{lstlisting}
public class Shape {
    public abstract void move(int dx, int dy);
}
\end{lstlisting}
\end{frame}


\begin{frame}[fragile]
\frametitle{Перемещение фигур}
\begin{lstlisting}
public class Line extends Shape {
    public void move(int dx, int dy) {
        x1 = x1 + dx;
        x2 = x2 + dx;
        y1 = y1 + dy;
        y2 = y2 + dy;
    }
}
\end{lstlisting}
\end{frame}


\begin{frame}[fragile]
\frametitle{Перемещение фигур}
\begin{lstlisting}
public class Rectangle extends Shape {
    public void move(int dx, int dy) {
        x1 = x1 + dx;
        x2 = x2 + dx;
        y1 = y1 + dy;
        y2 = y2 + dy;
    }
}
\end{lstlisting}
\end{frame}


\begin{frame}[fragile]
\frametitle{Перемещение фигур}
\begin{lstlisting}
public class Circle extends Shape {
    public void move(int dx, int dy) {
        x = x + dx;
        y = y + dy;
    }
}
\end{lstlisting}
\end{frame}


\begin{frame}[fragile]
\frametitle{Группировка фигур}
Добавим класс ShapeGroup, который будет представлять группу фигур.

\begin{lstlisting}
public class ShapeGroup extends Shape {
    public void addShape(Shape s);
    public void removeShape(Shape s);

    public void draw(SystemCanvas canvas);
    public void move(int dx, int dy);

    private ShapeList shapes;
}
\end{lstlisting}
\end{frame}


\begin{frame}[fragile]
\frametitle{Отрисовка группы фигур}
\begin{lstlisting}
public class ShapeGroup extends Shape {

    public void draw(SystemCanvas canvas) {
        ShapeListIterator it = shapes.getFirst();
        while(!it.isEnd()) {
            it.getShape().draw(graphics);
            it = it.getNext();
        }
    }

}
\end{lstlisting}
\end{frame}


\begin{frame}[fragile]
\frametitle{Перемещение группы фигур}
\begin{lstlisting}
public class ShapeGroup extends Shape {

    public void move(int dx, int dy) {
        ShapeListIterator it = shapes.getFirst();
        while(!it.isEnd()) {
            it.getShape().move(dx, dy);
            it = it.getNext();
        }
    }

}
\end{lstlisting}
\end{frame}


\begin{frame}[fragile]
\frametitle{Диаграмма классов}
\begin{center}
\includegraphics{shape_group.mps}
\end{center}
\end{frame}


\begin{frame}[fragile]
\frametitle{Паттерн Composite}
Назначение: компонует объекты в древовидные структуры. Позволяет клиентам
единообразно трактовать индивидуальные и составные объекты.
\end{frame}


\begin{frame}[fragile]
\frametitle{Участники}
\begin{itemize}
\end{itemize}
\end{frame}


\begin{frame}[fragile]
\frametitle{Паттерн Composite. Диаграмма классов}
\begin{center}
\includegraphics{composite.mps}
\end{center}
\end{frame}


\end{document}
